EFEK NEGATIF WIFI PADA KESEHATAN MANUSIA

1. Merusak Sel-Sel Pada Otak
   
   Organisasi kesehatan Dunia (WHO) telah mempelajari efek radisi non-termal pada kesehatan manusia yang dihasilkan oleh WiFi. 
   Hasilnya adalah paparan radiasi yang terjadi secara signifikan yang dapat meningkatkan resiko penyakit serius, terutama terjadi pada anak-anak
   Mengapa anak-anak sangat mudah terserang radiasi WiFi, karena anak-anak memiliki sistem saraf dan otak yang masih lemah atau
   masih dalam proses perkembangan. Dan mereka memiliki tengkorak yang sangat tipis dan lebih kecil dibandingkan dengan orang dewasa. 
   Dengan begitu, dapat berakibat lebih mudahnya terserang radiasi untuk menembus lebih kedalam otak meraka. Sehingga membuat sel-sel
   otak pada anak-anak menjadi rusak. Dan menimbulkan dampak yang tidak diinginkan seperti: penyebab sakit kepala, mudah terasa lelah, gangguan 
   konsentrasi, serta kehilangan memori dalam waktu yang pendek.

2. Menyebabkan Insomnia

   Pada tahun 2007, sebuah studi yang dilakukan mengevaluasi tentang modulasi frekuensi rendah dari beberapa perangkat yang menghasilkan
   gelombang elektromagnetik pada saat tidur. Hasil studi tersebut menunjukkan bahwah, mereka yang terkena radiasi elektromagnetik secara
   signifikan lebih sulit untuk tertidur serta mengalami perubahan pola gelombang otaknya. Hal ini pada akhirnya juga dapat mengembangkan 
   tingkat depresi maupun darah tinggi atau hipertensi yang diakibatkan karena kurangnya kualitas tidur yang baik. Untuk itu, para ahli kesehatan
   sangat menyarankan kita agar pada saat tidur jauhkanlah tubuh dari ponsel yang terhubung dengan wifi atau jaringan wifi yang sedang hidup.

3. Meningkatkan Resiko Masalah Perkembangan Pada Anak

   Dalam beberapa studi yang telah dilakukan oleh Organisasi Kesehatan Dunia (WHO), menyatakan bahwa paparan radiasi frekuensi radio non-termal
   dari WiFi atau telepon seluler dapat mengganggu perkembangan sel normal, terutama gangguan perkembangan pada janin. Dalam studi lain
   menyatakan bahwa, bahaya WiFi paada perkembangan anak mengakibatkan gangguan sintetis protein akibat paparan radiasi tersebut begitu parah terutama
   pada perkembangan anak-anak dan remaja.

4. Meningkatkan Resiko Masalah Infertilitas

   Dari beberapa penelitian telah menunjukkan bahwa pada paparan radiasi frekuensi WiFi dapat membawa efek negatif mempengaruhi sperma, hal tersebut
   dapat mengurangi gerakan sperma dan menyebabkan fragmentasi DNA. Selain itu, dari hasil studi yang dilakukan terhadap beberapa jenis hewan seperti
   pada tikus yang terkena frekuensi nirkabel selama dua jam sehari dalam kurun waktu 45 hari menunjukkan bahwa beberapa frekuensi nirkabel dapat 
   mencegah implantasi telur dan secara signifikan telah meningkatkan resiko stress oksidati. Akibat dari kerusakan sel yang berdampak pada struktur
   DNA dari paparan sinar radiasi tersebut menunjukkan kemungkinan kuat terjadinya kehamilan abnormal atau kegagalan telur untuk implan.

5. Meningkatkan Resiko Kanker

   Seperti yang telah kita ketahui bahwa, walaupun radiasi WiFi cukup rendah, akan tetapi jaringan WiFi bagi kesehatan manusia juga menghasilkan 
   gelombang yang sama. Dimana akibat radiasi dari gelombang tersebut dapat meningkatkan resiko seseorang untuk terjangkit penyakit yang sangat berbahaya
   misalnya, seperti: Kanker pada otak atau tumor pada otak.