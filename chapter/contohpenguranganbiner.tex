Jika digit desimal yang dikurangkan lebih kecil dari pada digit desimal yang akan dikurangi, maka yang terjadi adalah konsep peminjaman.
Digit tersebut akan meminjam satu dari digit sebelah kirinya.

Contoh 1

1111011            desimal 123
101001             desimal 41
——————— –
1010010            desimal 82

Pada contoh di atas tidak terjadi konsep peminjaman.

Contoh 2

0                  kolom ke 3 menjadi 0, sudah dipinjam
111101             desimal 61
10010              desimal 18
——–——— –
101011             Hasil pengurangan akhir 43


Pada soal yang kedua ini kita pinjam 1 dari kolom 3, karena ada selisih 0-1 pada kolom ke 2

Jika kita meminjam 1 dari kolom keenam untuk kolom kedua, karena kolom ketiga, keemat dan kolom kelima adalah nol. Setelah meminjam, kolom ketiga, keempat, dan kelima menjadi: 10 – 9 = 1
Hal ini juga berlaku dalam pengurangan biner, kecuali bahwa setelah meminjam kolom nol akan mengandung: 10 – 1 = 1

contoh 3 pengurangan bilangan biner 110001–1010 akan diperoleh hasil sebagai berikut:

1100101
10 10
——————— –
1001 11
